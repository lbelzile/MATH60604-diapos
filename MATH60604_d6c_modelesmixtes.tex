
\documentclass{beamer}
\usepackage{HECbeamer}
\usepackage{icomma}
\usepackage{numprint}
\title[\color{white}{MATH 60604 \S~6c - Modèles linéaires mixtes}]{\texorpdfstring{MATH 60604 \\Modélisation statistique \\ \S~6c - Modèles linéaires mixtes}{MATH 60604 \\Modélisation statistique \\ \S~6c - Modèles linéaires mixtes}}
\author{Léo Belzile}
\institute{HEC Montréal\\
Département de sciences de la décision}
\date{}

\begin{document}
\frame{\titlepage}

\begin{frame}
\frametitle{Introduction aux modèles à effets aléatoires}
Les modèles à effets aléatoires permettent d'inclure une corrélation intra-groupe et de faire des prédictions par groupe en plus d'inclure un effet de groupe au niveau de la population.
\bi \item La principale caractéristique des \alert{modèles linéaires mixtes} est de permettre l'inclusion d'\alert{effets aléatoires},  soit des paramètres qui varient d'un groupe à l'autre (ou d'une personne à l'autre pour les données répétées).
\item Bien que l'on permettre à chaque groupe d'avoir un effet différent, la moyenne de ces effets est nulle.
\ei
\end{frame}

\begin{frame}[fragile]
\frametitle{Modèles avec effets aléatoires}
\bi
% \item  Tel que mentionné plus tôt, les modèles mixtes permettent d'incorporer des effets aléatoires.
\item Lorsque une variable explicative est modélisée à l'aide d'un effet aléatoire, on suppose que \alert{l'effet total de cette variable est une combinaison de}:
\be
\item un \alert{effet commun à toute la population}
\item un \alert{effet propre aux sujets d'un groupe}.
\ee
\item Par exemple, dans le cas de mesures répétées sur des individus, l'effet d'une variable pourrait être formé d'un effet commun à tous les individus de la population et d'un effet unique et différent pour chaque individu.
\item Dans notre exemple sur la mobilisation des employés d'une unité, l'effet de l'ancienneté pourrait être formé d'un effet commun à tous les employés de l'entreprise et d'un effet unique et différent pour chaque unité.
\ei
\end{frame}

\begin{frame}
 \frametitle{Modèle linéaire mixte (formulation hiérarchique)}
Le modèle linéaire mixte s'écrit
 \begin{align*}
  \bs{Y}_i \mid \bs{\mathcal{B}}_i =\bs{b}_i &\sim \mathsf{No}_{n_i}\left( \mathbf{X}_i \bs{\beta} + \mathbf{Z}_i\bs{b}_i, \mathbf{R}_i\right) \\
  \bs{\mathcal{B}}_i & \sim \mathsf{No}_{q}( \bs{0}_q, \bs{\Omega})
 \end{align*}

 \bi \item 
 La réponse du groupe $i$, $\bs{Y}_i$ suit une loi normale conditionnellement aux \textbf{effets aléatoires} $  \bs{\mathcal{B}}_i$.
 \item 
On appelle désormais les coefficients $\bs{\beta}$ associés à la matrice du modèle $\mathbf{X}_i$ des \textbf{effets fixes}.
 \ei 
\end{frame}

\begin{frame}\frametitle{Modèle linéaire mixte: effets fixes}


On peut écrire le modèle linéaire mixte comme
\begin{align*}
[\bs{Y}_i \mid \bs{\mathcal{B}}_i=\bs{b}_i]= \mathbf{X}_i \bs{\beta} + \mathbf{Z}_i \bs{b}_i + \bs{\eps}_i, \qquad i = 1, \ldots, m.
\end{align*} 
où
\bi
\item $\bs{Y}_i=(Y_{i1},\ldots,Y_{in_i})^{\top}$ est le vecteur de taille $n_i$ contenant les réponses du groupe $i$.

\item $\mathbf{X}_i$ est la matrice $n_i \times (p+1)$ de variables explicatives du groupe $i$, dont la $i$e ligne est $\mathbf{X}_{ij}=(1,\mathrm{X}_{ij1},\ldots,\mathrm{X}_{ijp})^{\top}$. 
\bi 
\item La première colonne correspond à l'ordonnée à l'origine (vecteur de uns).
\item Les autres colonnes de $\mathbf{X}_i$ encodent chacune une variable explicative.
\ei
\item $\bs{\beta}$ est le $(p+1)$ vecteur d'effets \textbf{fixes}.
\ei
\end{frame}

\begin{frame}\frametitle{Modèles linéaires mixtes: effets aléatoires}
On peut écrire le modèle linéaire mixte comme
\begin{align*}
[\bs{Y}_i \mid \bs{\mathcal{B}}_i=\bs{b}_i]= \mathbf{X}_i \bs{\beta} + \mathbf{Z}_i b_i + \bs{\eps}_i, \qquad i = 1, \ldots, m.
\end{align*} 
où
\bi
\item $\mathbf{Z}_i$ est une matrice $n_i \times q$ qui contient un \textbf{sous-ensemble} des colonnes de $\mathbf{X}_i$. 
\bi \item Les colonnes de $\mathbf{Z}_i$ sont associées aux \textbf{effets aléatoires}. 
\item S'il n'y a pas d'effet aléatoire, $q=0$ et on recouvre le modèle linéaire usuel.
\ei
\item  $\bs{\mathcal{B}}_i=\bs{b}_i$ est un $q$ vecteur d'effets aléatoires propres au groupe $i$
\item $\bs{\eps}_i$ est un $n_i$ vecteur d'erreurs pour le groupe $i$.
\ei

\end{frame}

\begin{frame}
\frametitle{Forme générale pour les modèles mixtes}
Dans le modèle linéaire mixte, à la fois $\bs{\mathcal{B}}_i$ et $\bs{\eps}_i$ sont des vecteurs aléatoires. On suppose que 
\bi
\item les effets aléatoires $\bs{\mathcal{B}}_i$ et $\bs{\mathcal{B}}_j$ $(i \neq j)$ sont indépendants entre eux.
\item les effets aléatoires $\bs{\mathcal{B}}_i$ sont indépendants des erreurs $\bs{\eps}_j$.
\item les aléas $\bs{\eps}_i$ sont indépendants les uns des autres et ne dépendent pas des variables explicatives.
\item à la fois $\bs{\mathcal{B}}_i$ et $\bs{\eps}_i$ ont espérance zéro \[\E{\bs{\mathcal{B}}_i}=\bs{0}_{n_i}, \qquad \E{\bs{\eps}_i \mid \mathbf{X}_i}=\bs{0}_{n_i}\]
\ei
\end{frame}
\begin{frame}
\frametitle{Moyenne et variance conditionnelles et marginales}
On spécifie des modèles de covariance pour les effets aléatoires et les erreurs,
\begin{align*}
\Co{\bs{\mathcal{B}}_i}=\bs{\Omega}, \quad \Co{\bs{\eps}_i}=\mathbf{R}_i, \quad i=1, \ldots, m
\end{align*}
La moyenne et la variance \alert{conditionnelles} de $\bs{Y}_i$ sont
\begin{align*}
\E{\bs{Y}_i \mid \mathbf{X}_i,\bs{\mathcal{B}}_i=\bs{b}_i} = \mathbf{X}_i \bs{\beta} + \mathbf{Z}_i \bs{b}_i, \qquad \Co{\bs{Y}_i \mid \mathbf{X}_i, \bs{\mathcal{B}}_i=\bs{b}_i} = \mathbf{R}_i
\end{align*}
tandis que la moyenne et la variance \alert{marginales} de $\bs{Y}_i$ sont
\begin{align*}
\E{\bs{Y}_i \mid \mathbf{X}_i} = \mathbf{X}_i \bs{\beta}, \qquad \Co{\bs{Y}_i \mid \mathbf{X}_i} = \bs{\Sigma}_i=\mathbf{Z}_i \bs{\Omega}\mathbf{Z}_i^\top+\mathbf{R}_i.
\end{align*}
\end{frame}
\begin{frame}[fragile]
\frametitle{Paramètres d'un modèle mixte}
Les \alert{paramètres} du modèle à estimer sont
\bi \item le vecteur d'effets fixes, $\boldsymbol{\beta}$
\item les paramètres $\boldsymbol{\psi}$ de la covariance marginale $\bs{\Sigma}$ de $\bs{Y}$, qui découle de la structure de covariance  des aléas et des effets aléatoires.
\ei

\end{frame}
\begin{frame}
\frametitle{Effets groupes aléatoires}
\bi
\item Avec un modèle linéaire mixte,la moyenne conditionnelle $\E{Y_{ij}\mid \mathbf{X}_i, \bs{b}_i}$  peut être interprétée comme une \alert{prédiction} de la valeur de $Y_{ij}$ après avoir pris en compte les effets spécifiques à un groupe.
\item Quand on ajoute un effet aléatoire pour une variable groupe, on peut toujours estimer l'effet de variables fixes dans ce groupe.
\ei
\end{frame}

\begin{frame}{Prédiction}
\bi
\item On peut prédire $\bs{\mathcal{B}}_i$ par sa moyenne conditionnelle sachant $\bs{Y}_i$,
\begin{align*}
\E{\bs{\mathcal{B}}_i \mid  \bs{Y}_i}&=\bs{\Omega}\mathbf{Z}_i^\top \bs{\Sigma}_i^{-1}(\bs{Y}_i-\mathbf{X}_i\bs{\beta})
\shortintertext{où }
\bs{\Sigma}_i&=\mathbf{Z}_i \bs{\Omega}\mathbf{Z}_i^\top+\mathbf{R}_i.
\end{align*}

\item On peut remplacer ($\bs{\psi}, \bs{\beta})$ par leurs estimés $(\widehat{\bs{\psi}}, \widehat{\bs{\beta}})$ pour obtenir une prédiction de l'effet aléatoire
\begin{align*}
\hat{\bs{b}}_i= \hat{\bs{\Omega}}\mathbf{Z}_i^\top \hat{\bs{\Sigma}}_i^{-1}(\bs{Y}_i-\mathbf{X}_i\hat{\bs{\beta}})
\end{align*}
\ei
\end{frame}


\begin{frame}
 \frametitle{Effet fixe ou aléatoire?}
 \bi \item Il n'y a pas de consensus sur la définition d'effets fixes et aléatoires\ldots 
 \item De manière générale, la différence entre les deux est
 \bi 
%  \item At the population level, we assume that the average random effect is zero (or else captured by a fixed effect coefficient).
 \item les \textbf{effets fixes}  sont utilisés quand on a peu de groupes et beaucoup de valeurs au sein de chaque groupe. On s'intéresse à l'effet au sein du groupe (
 $m$ petit, $n_i$ grands).
 \item les \textbf{effets aléatoires} sont employés dans les cas de figure où il y a suffisamment de niveaux et de variabilité pour estimer de manière fiable la variance de l'effet aléatoire; on ne s'intéresse pas au niveau du groupe ($m$ grand, $n_i$ petit, variabilité).
 \ei
%  \item Tester si un effet aléatoire est présent revient à tester si la variable de l'effet aléatoire $\sigma^2_b=0$; c'est un test d'hypothèse pour lequel la loi nulle n'est pas standard \ldots 
 \ei
\end{frame}

\end{document}
