\documentclass{beamer}
\usepackage{HECbeamer}
\usepackage{icomma}
\usepackage{numprint}
\title[\color{white}{MATH 60604 \S~5g - Comparaison de structures de covariance}]{\texorpdfstring{MATH 60604 \\Modélisation statistique \\ \S~5g - Comparaison de structures de covariance}{MATH 60604 \\Modélisation statistique \\ \S~5g - Comparaison de structures de covariance}}
\author{}
\institute{HEC Montréal\\
Département de sciences de la décision}
\date{} 

\begin{document}
\frame{\titlepage}

\begin{frame}[fragile]
\frametitle{Résumé des modèles de covariance traités}
\bi
\item Nous avons vu comment ajuster cinq types de structure de covariance sur les résidus du modèle de régression:
\bi

\item Une structure d'équicorrélation (\code{cs}) (échangeable): toutes les paires d'observations ont la même corrélation.
\item Une structure \alert{$\mathsf{AR}(1)$}: la corrélation entre deux observations décroit géométriquement avec le temps écoulé entre les deux. 
\item Une structure \alert{$\mathsf{ARH}(1)$}: une extension hétérogène du modèle $\mathsf{AR}(1)$, avec la même corrélation mais des variances différentes pour chaque temps.
\item Une covariance \alert{non structurée}, qui permet une covariance différente pour chaque paire d'observations dans le temps.
\ei
\ei
\end{frame}

\begin{frame}[fragile]
\frametitle{Sélection de la structure de covariance}
\begin{center}
\begin{footnotesize}
\begin{tabular}{l c c c}
\toprule 
\textbf{Modèle} & $-2\ell_{\textrm{reml}}$ & \textbf{AIC} & \textbf{BIC} \\ \midrule
Indépendance & $ \numprint{776.7}$ & $ \numprint{778.7}$ & $\numprint{782.6}$ \\ 
Équicorrélation & $\numprint{709.4}$ & $ \numprint{713.4}$ & $\numprint{718.2}$ \\ 
$\mathsf{AR}(1)$ & $\numprint{681.8}$ & \alert{$\numprint{685.8}$} & \alert{$ \numprint{690.5}$} \\ 
$\mathsf{ARH}(1)$ & $ \numprint{675.3}$ & $ \numprint{687.3}$ & $ \numprint{701.6}$ \\ 
Non structurée & $ \numprint{659.3}$ & $ \numprint{689.3}$ & $ \numprint{725.0}$ \\ \bottomrule
\end{tabular}
\end{footnotesize}
\end{center}
\bi
\item  Les critères d'information \AIC{} et \BIC{} mène au choix du modèle avec la structure $\mathsf{AR}(1)$. 
\item Il s'agit d'un modèle parcimonieux (deux paramètres pour la structure de covariance) qui semble bien tenir compte de la corrélation intra-sujet.
\item Ce qui est rassurant en plus, c'est que peu importe la structure utilisée, les conclusions quant aux effets des variables explicatives sont toujours les mêmes.
\ei
\end{frame}


\begin{frame}
\frametitle{Choix de la structure de covariance}
\bi
\item Petit détail technique: si on veut comparer des modèles
avec le \AIC{} ou le \BIC{} et que la méthode d'estimation REML est utilisée, il faut
absolument que les modèles comparés \alert{contiennent les mêmes variables explicatives} (effets fixes).
\item Les critères d'information provenant de
deux modèles ayant des effets fixes différents, estimés par REML, ne sont \textbf{pas} comparables. Par contre, ils sont comparables si on utilise la méthode du maximum de vraisemblance.
\ei
\end{frame}







\begin{frame}[fragile]
\frametitle{Remarques importantes sur les tests d'hypothèses sur la structure de covariance}
\bi
\item La plupart du temps, les questions de recherche font référence à des tests d'hypothèses concernant les paramètres de la partie ``moyenne'' du modèle. 
\item Dans le cas de données corrélées, nous savons maintenant qu'il faut que la partie
covariance soit modélisée adéquatement afin que l'inférence sur la partie
moyenne soit valide. 
\item Procéder en choisissant la structure de covariance selon des critères comme
le AIC et le BIC est raisonnable. Mais il est aussi possible de faire des
tests formels sur les paramètres de la structure de covariance.
\ei
\end{frame}

\begin{frame}
\frametitle{Rappel sur les tests de rapport de vraisemblance (REML)}

\bi
\item Le test compare la log-vraisemblance du modèle complet (sous $\Hy_1$) à celle du modèle restreint emboîté (sous $\Hy_0$).
\item L'hypothèse nulle est que le modèle restreint est une simplification adéquate du modèle complet.
\item La statistique du test de rapport de vraisemblance est
\begin{align*}
D=2\{\ell_{\textrm{reml}}(\hat{\bs{\theta}})- \ell_{\textrm{reml}}(\hat{\bs{\theta}}_0)\} \end{align*}
\item Sous $\Hy_0$, $D \stackrel{\cdot}{\sim}\chi^2_k$, où le nombre de degrés de liberté $k$ est la différence du nombre de paramètres des deux modèles. 
\item On calcule la valeur-$p$ du test à partir de la loi $\chi^2_k$.
\ei
\end{frame}
\begin{frame}[fragile]
\frametitle{Tests du rapport de vraisemblance pour structures de covariance}
\bi
\item Il est possible de tester des hypothèses compliquées à l'aide du test de rapport de vraisemblance.
\item Par exemple, on peut vérifier s'il est nécessaire de spécifier des variances différentes avec une structure autorégressive $\mathsf{AR}(1)$.
\item Dans ce cas, on veut tester si le modèle $\mathsf{AR}(1)$ est adéquat ou si le modèle $\mathsf{ARH}(1)$ est préférable.
\item L'hypothèse nulle est $
\Hy_0: \sigma_1^2=\sigma_2^2=\cdots=\sigma_5^2$ contre l'alternative qu'au moins deux des variances sont différentes.
\item Le \alert{modèle complet} est le modèle $\mathsf{ARH}(1)$ ($\Hy_1$) et le \alert{modèle restreint} le modèle $\mathsf{AR}(1)$ ($\Hy_0$).
\ei
\end{frame}

\begin{frame}[fragile]
\frametitle{Statistique du rapport de vraisemblance: $\mathsf{AR}(1)$ versus $\mathsf{ARH}(1)$}
\bi
\item Sur la base des sorties précédentes, la différence de log-vraisemblance restreinte $-2\ell_{\textrm{reml}}$ pour ces deux modèles est $681.8-675.3=6.5$.
\item Il y a \textbf{quatre} paramètres additionnels dans le modèle complet. 

On compare la valeur de la statistique $95$\% quantile d'une variable $\chi^2_4$, soit $9.48$.

\begin{tcolorbox}[colback=white,colframe=hecblue,title=Code \SASlang{} pour calculer la valeur-$p$ avec la loi nulle $\chi^2_4$]
\begin{verbatim}
data pval;
pval=1-CDF('CHISQ', 6.5, 4);
run;
proc print data=pval;
run;
\end{verbatim}
\end{tcolorbox}
\item On obtient une valeur-$p$ de $0.165$. On ne rejette pas $\Hy_0$ --- le modèle $\mathsf{AR}(1)$ est une simplification adéquate du modèle $\mathsf{ARH}(1)$.
\ei
\end{frame}

\begin{frame}
\frametitle{Remarques finales}
\bi
\item On a utilisé la méthode du maximum de vraisemblance restreinte (REML) pour estimer les paramètres de variance (option par défaut de \code{proc mixed}). 
\item Plusieurs modèles considérés sont emboîtés, donc on peut faire des tests pour établir des comparaisons.
\bi \item par exemple, l'indépendance $\prec$ $\mathsf{AR}(1)$$\prec$ $\mathsf{ARH}(1)$ $\prec$ non structurée.
\ei
\item L'utilisation des critères \AIC{} et \BIC{} pour comparer ces modèles est valide \textbf{tant que} la structure de la moyenne inclut les \textbf{même} variables explicatives, comme ce fut le cas dans tous les modèles ajustés dans ces diapositives.
\item Si on veut comparer des modèles qui incluent des variables explicatives différentes, il faudrait utiliser la méthode du maximum de vraisemblance et non pas la méthode REML.
\ei
\end{frame}
\end{document}
