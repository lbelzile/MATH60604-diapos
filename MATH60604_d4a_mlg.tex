\documentclass{beamer}
\usepackage{HECbeamer}
\usepackage{icomma}
% \usepackage{pgfpages}
% \pgfpagesuselayout{4 on 1}[letterpaper, landscape, border shrink=5mm]
\title[\color{white}{MATH 60604 \S~4a - Modèles linéaires généralisés}]{\texorpdfstring{MATH 60604 \\Modélisation statistique \\ \S~4a - Modèles linéaires généralisés}{MATH 60604 \\Modélisation statistique \\ \S~4a - Modèles linéaires généralisés}}
\author{}
\institute{HEC Montréal\\
Département de sciences de la décision}
\date{} 

\begin{document}
\frame{\titlepage}
\begin{frame}
\frametitle{Introduction}
\bi
\item Les modèles linéaires ne sont adéquats que pour les variables réponses qui suivent (conditionnellement) une loi normale.

\item Dans plusieurs contextes, les variables réponse $Y$ à disposition sont
\bi
\item \alert{binaires}, 
\item entières, c'est-à-dire des variables de \alert{dénombrement}, 
\item \alert{continues, mais non-négatives}, 
\ei
\item On considère des lois adéquates pour des données binaires, des proportions et des variables de dénombrement, afin de faire de l'inférence basée sur la vraisemblance.
\ei
\end{frame}


\begin{frame}
\frametitle{Variables réponses binaires}
\bi
\item Si la variable réponse $Y$ vaut soit $0$, soit $1$, on peut postuler une loi \alert{Bernoulli} pour $Y$, soit
\begin{align*}
\P{Y=y} = \pi^y (1-\pi)^{1-y}, \quad y=0, 1.
\end{align*}
Il en découle que $\E{Y}=\pi$ et $\Va{Y}=\pi(1-\pi)$.
\item Par convention, les zéros représentent des échecs (non) et les uns des réussites (oui).
\item Exemples de questions de recherches comprenant une variable réponse binaire:
\bi
\item est-ce qu'un client potentiel a répondu favorablement à une offre
promotionnelle?
\item est-ce qu'un client est satisfait du service après-vente?
\item est-ce qu'une firme va faire faillite au cours des trois prochaines années?
\item est-ce qu'un participant à une étude réussit une tâche?
\ei
\ei
\end{frame}

\begin{frame}
\frametitle{Variables réponses binaires cumulées}
Si les données représentent la somme d'événements Bernoulli indépendants, la loi du nombre de réussites $Y$ sur le nombre d'essais $m$ est binomiale avec fonction de masse
\begin{align*}
\P{Y=y} = \binom{m}{y}\pi^y (1-\pi)^{m-y}, \quad y=0, 1.
\end{align*}
La vraisemblance pour un échantillon $\mathsf{Bin}(m, \pi)$  est (à constante de normalisation près qui ne dépend pas de $\pi$) la même que pour un échantillon aléatoire de $m$ variables Bernoulli indépendantes. L'espérance est $\E{Y}=m\pi$ et la variance $\Va{Y}=m\pi(1-\pi)$. 
\end{frame}
\begin{frame}
\frametitle{Variables de dénombrement}
\bi
\item \alert{Si la probabilité d'un événement est \textbf{rare}}, on suppose souvent que $Y$, le nombre de réussites dans un intervalle de temps donné, suit une loi de \alert{Poisson},
\begin{align*}
\P{Y=y} = \frac{\exp(-\mu)\mu^y}{y!}, \quad y=0, 1, 2, \ldots
\end{align*}
\item Le paramètre $\mu$ de la loi de Poisson est à la fois l'espérance et la variance de la variable,  $\E{Y}=\Va{Y}=\mu$.
\item Exemples de questions de recherches comprenant une variable réponse de dénombrement: 
\bi
\item nombre de réclamations faites par un client d'une compagnie d'assurance
au cours d'une année.
\item  nombre d'achats effectués par un client depuis un mois.
\item nombre de tâches réussies par un participant lors d'une étude.
\ei
\ei
\end{frame} 
% \begin{frame}
% \frametitle{Modèles linéaires généralisés}
% \bi
% \item Les modèles linéaires
% généralisés permettent de
% modéliser adéquatement des variables réponses quelconques (binaires, de dénombrement, positive, continues) en utilisant des lois aléatoires appropriées. 
% \bi \item La régression
% linéaire (avec des erreurs normales) est un \alert{cas particulier} des modèles linéaires généralisés.
% \ei
% \item Dans ce chapitre, nous allons présenter une introduction aux modèles
% linéaires généralisés, en portant un intérêt particulier à la régression logistique et Poisson. 
% \item Nous nous limiterons au cas où les observations sont
% indépendantes. 
% \bi 
% \item Le traitement des données corrélées et longitudinales pour des modèles linéaire généralisés (\alert{modèles linéaires mixtes généralisés}) sont couvertes dans le cours MATH80621.
% \ei
% \ei 
% \end{frame}


\begin{frame}
\frametitle{Notation pour les modèles linéaires généralisés}
\bi
\item Le point de départ est le même que pour la régression linéaire:
\bi
 
\item On dispose
d'un échantillon d'observations indépendantes
\begin{align*}
(Y_i, \mathrm{X}_{i1}, \ldots, \mathrm{X}_{ip}), \quad i=1, \ldots, n
\end{align*}
où $Y$ est la variable réponse et $\mathrm{X}_1, \ldots, \mathrm{X}_p$ 
des variables explicatives supposées connues et fixes (non-aléatoires).
\item Le but est modéliser la variable réponse en fonction des variables explicatives.
\ei
\item On dénote $\mu_i$ la \alert{moyenne} (conditionnelle) de $Y_i$ étant données les variables explicatives,
\begin{align*}
\mu_i=\E{Y_i \mid  \mathrm{X}_{i1}, \ldots, \mathrm{X}_{ip}}.
\end{align*}
\item On dénote par $\eta_i$ la \alert{combinaison linéaire} des variables explicatives qui servira à modéliser la
variable réponse, 
\begin{align*}
\eta_i=\beta_0 + \beta_1 \mathrm{X}_{i1} + \cdots + \beta_p \mathrm{X}_{ip}.
\end{align*}
\ei
\end{frame}


\begin{frame}
\frametitle{Composantes du modèle linéaire généralisé}
Trois composantes sont nécessaires pour définir un modèle linéaire
généralisé
\bi

\item Une loi de probabilité pour la variable réponse $Y$ qui fait partie de la famille exponentielle (normale, binomiale, Poisson, gamma, \ldots). 
\item Un prédicteur linéaire $\boldsymbol{\eta} = \mathbf{X}\bbeta$.
\item Une fonction monotone $g$, appelée \alert{fonction de liaison}, que \alert{relie} la moyenne de $Y_i$ aux variables explicatives, $g(\mu_i)=\eta_i$, d'où
\begin{small}
\begin{align*}
g(\mu_i) & = \eta_i = \beta_0 + \beta_1 \mathrm{X}_{i1} + \cdots + \beta_p \mathrm{X}_{ip} \\
\quad \Leftrightarrow \quad \mu_i & = g^{-1}(\eta_i) = g^{-1}(\beta_0 + \beta_1 \mathrm{X}_{i1} + \cdots + \beta_p \mathrm{X}_{ip}).
\end{align*}
\end{small}
\ei
\end{frame}
\begin{frame}[fragile]
\frametitle{Fonction de liaison}
\bi
\item Dans le modèle de régression linéaire ordinaire, on n'impose pas de contraintes aux valeurs prises par la moyenne $\mu_i$ et $\hat{\mu}_i = \widehat{\eta}_i$ peut prendre des valeurs arbitraires dans l'intervalle $(-\infty, \infty)$. 
\item En revanche, les moyennes de certaines variables réponses sont contraintes
\bi
 
\item variables Bernoulli/binomiales: la moyenne $\mu=\pi$ doit être dans l'intervalle $(0, 1)$.
\item variables Poisson: la moyenne $\mu$ doit être positive.
\ei
\item Un choix adéquat de fonction de liaison pour $\mu_i$ permet une transformation de la combinaison linéaire $\eta_i$ de telle sorte qu'aucune contrainte numérique n'est imposée sur les paramètres $\bs{\beta}$.
\ei
\end{frame}

\begin{frame}
\frametitle{Choix de la fonction de liaison}
Certains choix de fonctions de liaisons facilitent l'interprétation des paramètres ou l'optimisation avec la fonction de vraisemblance.
\bi
\item Pour les lois Bernoulli et binomiale, la fonction de liaison la plus utilisée est la fonction  \alert{logit}, 
\begin{align*}
\logit(\mu)\coloneqq  
\ln \left( \frac{\mu}{1-\mu} \right) = \eta \quad \Leftrightarrow \quad \mu = \frac{\exp(\eta)}{1+\exp(\eta)}.
\end{align*}

\item Pour la loi Poisson, la fonction de liaison canonique est le \alert{logarithme naturel},
\begin{align*}
\ln (\mu) = \eta  \quad \Leftrightarrow \quad \mu = \exp(\eta).                                                  \end{align*}

\item Pour la loi normale, la fonction de liaison est la fonction \alert{identité}, donc $\mu = \eta$.
% In particular, the \textbf{canonical link} function is such that $\theta = \eta$.
\ei
\end{frame}

\begin{frame}
\frametitle{Cas particulier de la régression linéaire}
\bi
\item La régression linéaire ordinaire est un cas spécial de régression linéaire généralisée, avec
\begin{align*}
Y_i = \beta_0+ \beta_1 \mathrm{X}_{i1}+\ldots+\beta_p \mathrm{X}_{ip} + \varepsilon_i, \qquad (i=1, \ldots, n)
\end{align*}
où $\varepsilon_i \simiid \mathsf{No}(0, \sigma^2)$, c'est-à-dire que $\eps_1, \ldots, \eps_n$ sont des variables indépendantes et identiquement distribuées de loi normale avec moyenne $0$ et variance $\sigma^2$.
\item De façon équivalente, 
\begin{align*}
Y_i\mid \mathbf{X}_i \stackrel{\mathsf{ind}}{\sim} \mathsf{No}(\beta_0+ \beta_1 \mathrm{X}_{i1}+\ldots+\beta_p \mathrm{X}_{ip}, \sigma^2)
\end{align*}
% \item Thus, we have:
% \begin{align*}
% \E{Y_i\mid \mathbf{X}_i}=\beta_0+ \beta_1 \mathrm{X}_{i1} + \cdots + \beta_p \mathrm{X}_{ip}.
% \end{align*}
\item On déduit que la régression linéaire est un modèle linéaire généralisé avec
\bi
 
\item une loi normale pour la réponse et
\item la fonction identité comme fonction de liaison.
\ei 
\ei
\end{frame} 
\end{document}



